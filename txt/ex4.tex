\documentclass{jarticle}
\usepackage{amsmath}
\usepackage{listings}
\lstset{
  basicstyle={\ttfamily},
  frame={tb}
}

\renewcommand{\lstlistingname}{プログラム}

\title{計算機システム演習 第4回レポート}
\author{17B13541 \and 細木隆豊}
\date{}

\begin{document}
\maketitle
  \section{課題1の議論}
    caller-saveでは、サブルーチンをループによって計8回呼び出すのでレジスタの退避・復帰を8回行う。

    callee-saveでは、今回のサブルーチンread1bitで\$s0, \$s1は使われないため、read1bitでレジスタを退避・復帰しなくて良い。mainルーチンでは\$s0, \$s1を使うので、退避・復帰をする。よって、callee-saveでの退避・復帰は1回である。

    以上から、今回の2進10進変換プログラムではレジスタの退避・復帰回数はcaller-saveとcallee-saveで大きく異なり、callee-saveを方が効率的である。
  \section{説明・工夫}
    \begin{lstlisting}[caption=assignment1.s]
    .text
main:
    addi  $sp, $sp, -12     # レジスタの退避領域作成
    sw    $ra, 0($sp)       # $ra を退避
    sw    $s0, 4($sp)       # $s0 を退避
    sw    $s1, 8($sp)       # $s1 を退避

    li    $s0, 0            # $s0 に初期値を設定
                            # $s0 は結果の記録
    li    $s1, 0            # $s1 に初期値を設定
                            # $s1 はループ回数の記録
loop:
    jal   read1bit          # サブルーチンの呼び出し

    add   $s0, $s0, $s0     # 今までの結果を2倍
    add   $s0, $s0, $v0     # 入力値を足す
    addi  $s1, $s1, 1       # ループ回数に1足す
    blt   $s1, 8, loop      # ループ回数を比較

    move  $a0, $s0
    li    $v0, 1
    syscall                 # 結果を表示

    lw    $ra, 0($sp)       # $ra を復帰
    lw    $s0, 4($sp)       # $s0 を復帰
    lw    $s1, 8($sp)       # $s1 を復帰
    addi  $sp, $sp, 12      # 退避領域開放
    jr    $ra               # main ルーチン終了

read1bit:
    li    $v0, 5
    syscall                 # 入力値を読み込み
    jr    $ra               # サブルーチン終了
    \end{lstlisting}

    \begin{lstlisting}[caption=assignment2.s]
    .data                   # data の設定
cnt:                        # 配列の要素数を入力させる
    .asciiz "What is number of the array?-"
array:                      # 配列の要素を入力させる
    .asciiz "type elements of the array.\n"
sum:                        # 配列の要素の和を出力
    .asciiz " is the sum."

    .text
main:
          ~ 退避の部分により省略 ~

    li    $v0, 4
    la    $a0, cnt
    syscall                 # 文字列 cnt を表示

    li    $v0, 5
    syscall
    move  $s0, $v0          # 配列の要素数を読み込み

    li    $v0, 4
    la    $a0, array
    syscall                 # 文字列 array を表示

    move  $a0, $s0          # サブルーチンの引数を設定
    jal   create_array      # $a0 は配列の要素数

    move  $a0, $v0          # create_array の戻り値を
                            # calc_sum の引数に設定
                            # $a0 は配列の先頭アドレス
    move  $a1, $s0          # 引数を設定
    jal   calc_sum          # $a1 は配列の要素数
    move  $s0, $v0          # calc_sum の結果を代入

    li    $v0, 1
    move  $a0, $s0
    syscall                 # 配列の和を表示

    li    $v0, 4
    la    $a0, sum
    syscall                 # 文字列 sum を表示

          ~ 復帰の部分により省略 ~

create_array:
          ~ 退避の部分により省略 ~
    move  $s0, $a0          # $s0 は残りのループ回数を記録

    li    $v0, 9
    add   $a0, $a0, $a0
    add   $a0, $a0, $a0     # 要素数 * 4 Byte
    syscall                 # メモリの確保
    move  $s1, $v0          # メモリの先頭アドレス

    sw    $v0, 8($sp)       # 先頭アドレスを記憶
loopa:
    li    $v0, 5
    syscall                 # 入力値を読み込み
    sw    $v0, 0($s1)       # 入力値を配列に追加

    addi  $s1, $s1, 4       # 次の要素が入るメモリ
    addi  $s0, $s0, -1      # 残りのループ回数を減らす
    blt   $zero, $s0, loopa # ループするか判定

          ~ 復帰の部分により省略 ~
    jr    $ra

calc_sum:
          ~ 退避の部分により省略 ~

    li    $s0, 0            # $s0 は配列の要素を読み込む
    li    $s1, 0            # $s1 はそれまでの和
loopb:
    lw    $s0, 0($a0)       # $s0 に要素を呼び出す
    add   $s1, $s1, $s0     # $s0 に $s1 を足す
    addi  $a0, $a0, 4       # 次の要素があるアドレス
    addi  $a1, $a1, -1      # 残りのループ回数
    blt   $zero, $a1, loopb # ループするか判定

    move  $v0, $s1          # 戻り値に結果を代入
          ~ 復帰の部分により省略 ~
    jr    $ra
    \end{lstlisting}
    課題2はcallee-saveで実装しました。
  \section{実行結果}
    assignment1.s
    \begin{lstlisting}
      1
      1
      0
      0
      1
      1
      0
      0
      204
    \end{lstlisting}
    \begin{lstlisting}
      1
      0
      0
      0
      0
      0
      0
      0
      128
    \end{lstlisting}

    assignment2.s
    \begin{lstlisting}
      What is number of the array?-5
      type elements of the array.
      1
      2
      3
      4
      5
      15 is the sum.
    \end{lstlisting}
    \begin{lstlisting}
      What is number of the array?-10
      type elements of the array.
      1
      2
      3
      4
      5
      6
      7
      8
      9
      10
      55 is the sum.
    \end{lstlisting}
  \section{感想・質問}
  難しかったが、理解できたので大変良かった。
  
  はじめ勘違いをしておりループの事を考えていなかったので、課題1の議論を間違えてしまいました。理解したうえで修正しました。
\end{document}
