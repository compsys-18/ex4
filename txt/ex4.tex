\documentclass{jarticle}
\usepackage{amsmath}
\usepackage{listings}
\lstset{
  basicstyle={\ttfamily\normalsize},
}

\renewcommand{\lstlistingname}{プログラム}

\title{計算機システム演習 第4回レポート}
\author{17B13541 \and 細木隆豊}
\date{}

\begin{document}
\maketitle
  \section{課題1の議論}
    caller-saveでは、サブルーチンを1回呼び出すのでレジスタの退避・復帰を1回行う。

    callee-saveでは、今回のサブルーチンread1bitで\$s0, \$s1は使われないため、read1bitでレジスタを退避・復帰しなくて良い。mainルーチンでは\$s0, \$s1を使うので、退避・復帰をする。よって、callee-saveでの退避・復帰は1回である。

    以上から、今回の2進10進変換プログラムではレジスタの退避・復帰回数はどちらも1回であるから、コードの効率の違いは生まれないと考えられる。
  \section{説明・工夫}
    \begin{lstlisting}[caption=assignment1.s]
    .text
main:
    addi  $sp, $sp, -12     # レジスタの退避領域作成
    sw    $ra, 0($sp)       # $ra を退避
    sw    $s0, 4($sp)       # $s0 を退避
    sw    $s1, 8($sp)       # $s1 を退避

    li    $s0, 0            # $s0 に初期値を設定
                            # $s0 は結果の記録
      li    $s1, 0          # $s1 に初期値を設定
                            # $s1 はループ回数の記録
loop:
    jal   read1bit          # サブルーチンの呼び出し

    add   $s0, $s0, $s0     # 今までの結果を2倍
    add   $s0, $s0, $v0     # 入力値を足す
    addi  $s1, $s1, 1       # ループ回数に1足す
    blt   $s1, 8, loop      # ループ回数を比較

    move  $a0, $s0
    li    $v0, 1
    syscall                 # 結果を表示

    lw    $ra, 0($sp)       # $ra を復帰
    lw    $s0, 4($sp)       # $s0 を復帰
    lw    $s1, 8($sp)       # $s1 を復帰
    addi  $sp, $sp, 12      # 退避領域開放
    jr    $ra               # main ルーチン終了

read1bit:
    li    $v0, 5
    syscall                 # 入力値を読み込み
    jr    $ra               # サブルーチン終了
    \end{lstlisting}
    \begin{lstlisting}[caption=assignment2.s]
    .data                   # data の設定
cnt:                        # 配列の要素数を入力させる
    .asciiz "What is number of the array?-"
array:                      # 配列の要素を入力させる
    .asciiz "type elements of the array.\n"
sum:                        # 配列の要素の和を出力
    .asciiz " is the sum."

    .text
main:
    addi  $sp, $sp, -8      # レジスタの退避領域作成
    sw    $ra, 0($sp)       # $ra を退避
    sw    $s0, 4($sp)       # $s0 を退避

    li    $s0, 0            # $s0 に初期値を設定

    li    $v0, 4
    la    $a0, cnt
    syscall                 # 文字列 cnt を表示

    li    $v0, 5
    syscall
    move  $s0, $v0          # 配列の要素数を読み込み

    li    $v0, 4
    la    $a0, array
    syscall                 # 文字列 array を表示

    move  $a0, $s0          # サブルーチンの引数を設定
    jal   create_array      # サブルーチンの呼び出し

    move  $a0, $v0
    move  $a1, $s0
    jal   calc_sum
    move  $s0, $v0

    li    $v0, 1
    move  $a0, $s0
    syscall

    li    $v0, 4
    la    $a0, sum
    syscall

    lw    $s0, 4($sp)
    lw    $ra, 0($sp)
    addi  $sp, $sp, 8
    jr    $ra

create_array:
    addi  $sp, $sp, -12
    sw    $s0, 0($sp)
    sw    $s1, 4($sp)

    move  $s0, $a0

    li    $v0, 9
    add   $a0, $a0, $a0
    add   $a0, $a0, $a0
    syscall
    move  $s1, $v0

    sw    $v0, 8($sp)
loopa:
    li    $v0, 5
    syscall
    sw    $v0, 0($s1)

    addi  $s1, $s1, 4
    addi  $s0, $s0, -1
    bne   $s0, $zero, loopa

    lw    $v0, 8($sp)
    lw    $s1, 4($sp)
    lw    $s0, 0($sp)
    addi  $sp, $sp, 12
    jr    $ra

calc_sum:
    addi  $sp, $sp, -8
    sw    $s0, 0($sp)
    sw    $s1, 4($sp)

    li    $s0, 0
    li    $s1, 0
loopb:
    lw    $s0, 0($a0)
    add   $s1, $s1, $s0
    addi  $a0, $a0, 4
    addi  $a1, $a1, -1
    bne   $a1, $zero, loopb

    move  $v0, $s1
    lw    $s1, 4($sp)
    lw    $s0, 0($sp)
    addi  $sp, $sp, 8
    jr    $ra
    \end{lstlisting}
  \section{実行結果}
  \section{感想・質問}
\end{document}
